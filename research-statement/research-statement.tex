\documentclass[11pt,letterpaper]{article}
\usepackage[margin=0.75in]{geometry}
\usepackage{graphicx}
%\usepackage[bf,tiny,compact]{titlesec}
\usepackage{times}
\usepackage{cite}
\usepackage{natbib}
\usepackage{titlesec}
\usepackage{etoolbox}
\usepackage{hyperref}


\makeatletter
\def\@maketitle{%
    {\centering\fontsize{12pt}{14pt}\selectfont\textbf{\@title}\par}
}
\patchcmd{\@footnotetext}{\footnotesize}{\fontsize{10pt}{12pt}\selectfont}{}{}
\makeatother


\renewcommand{\bibsection}{\noindent\textbf{References (works to which I contrubted have my name \underline{underlined})}\vspace{-6pt}}
\setlength{\bibsep}{0pt} % or use whatever dimension you want
\renewcommand{\bibfont}{\fontsize{10pt}{12pt}\selectfont} % or any other  appropriate font command


\titleformat{\section}{\normalfont\large\bfseries\scshape}{}{0em}{}
\titlespacing*{\section}{0em}{0.5\baselineskip}{0em}
\titleformat{\subsection}[runin]{\normalfont\normalsize\bfseries}{}{0em}{}[.]
\titlespacing*{\subsection}{0em}{0.25\baselineskip}{0.5em}


%% LOGIC FOR CUSTOMIZING THE STATEMENT FOR INDIVIDUAL SCHOOLS
% set up common defines (commands and boolean flags)
% Sets up custom commands and flags for school-specific behaviors
% Also assigns default values, but these can be overwritten with school-specific
% values in the corresponding header file.
% Some default values are placeholders (marked in red) and these MUST be 
% overwritten in the school-specific headers!
\newcommand{\schoolname}{\textcolor{red}{COLLEGE\_NAME}\xspace} 
\newcommand{\schoolnamelong}{\textcolor{red}{COLLEGE\_NAME\_LONG}\xspace}
\newcommand{\schooladdress}{\textcolor{red}{47 Generic St., City, ST 99999}\xspace}
\newcommand{\schoolintrocourses}{\textcolor{red}{[school-specific intro course numbers]}\xspace}
\newcommand{\schooladvcourses}{\textcolor{red}{[school-specific AI/ML/NLP course numbers]}\xspace}
\newcommand{\chairname}{\textcolor{red}{Prof. Search Chair}\xspace}
\newcommand{\chairlastname}{\textcolor{red}{Prof. Chair}\xspace}
\newcommand{\chairtitle}{\textcolor{red}{Chair, Faculty Search Committee}\xspace}
\newcommand{\department}{Department of Computer Science\xspace}
\newcommand{\position}{\textcolor{red}{POSITION\_NAME (POSITION\_ID)}\xspace}

\newif\iflongdei
\longdeitrue 
\newif\ifappendix
\appendixtrue 
\newif\ifliberalarts
\liberalartstrue
\newif\ifsupplementals
\supplementalsfalse

\newcommand{\materials}{my latest CV, a graduate transcript, and statements on teaching, research, and commitment to diversity and inclusion.\xspace}

\newcommand{\coverteachingpara}{%
\textcolor{red}{[school-specific segment on teaching philosophy]}
}

\newcommand{\coverresearchpara}{%
\textcolor{red}{[school-specific segment on research]}
}

\newcommand{\lateachingintro}{%
Liberal Arts education played a formative role in my approach to teaching: my love of teaching was first sparked during my undergraduate studies at Harvey Mudd College.
That was where I got my first (small) taste of what it is like to be an instructor, through my participation in the CS department's ``grutoring'' program---similar to TA positions found at larger universities, but with more emphasis on direct interaction with students.
This early experience of engaging directly with individual students, and seeing the specific things they struggled with, strongly influenced how I think about teaching.
To this day, I approach teaching by starting from a student's perspective: for any given course, what is a student most likely to struggle with, and what's the most effective way to help them understand it?
While the specific answers to these questions will vary with each course and each student cohort, there are nevertheless a few concrete strategies which I have found to be good \emph{starting points} for any course: \textbf{teaching with narratives}, \textbf{leveraging technology for interactivity}, and \textbf{building inclusive learning environments}.%
}

\newcommand{\genteachingintro}{%
My love of teaching was first sparked during my undergraduate studies at Harvey Mudd College.
That was where I got my first (small) taste of what it is like to be an instructor, through my participation in the CS department's ``grutoring'' program---similar to TA positions found at larger universities, but with more emphasis on direct interaction with students.
This early experience of engaging directly with individual students, and seeing the specific things they struggled with, strongly influenced how I think about teaching.
Even as I have moved on to teaching in larger classroom settings during my graduate career, I continue to approach teaching by starting from a student's perspective: for any given course, what is a student most likely to struggle with, and what's the most effective way to help them understand it?
While the specific answers to these questions will vary with each course and each student cohort, there are nevertheless a few concrete strategies which I have found to be good \emph{starting points} for any course: \textbf{teaching with narratives}, \textbf{leveraging technology for interactivity}, and \textbf{building inclusive learning environments}.%
}

\newcommand{\lanarrativeend}{%
This makes it particularly valuable for a Liberal Arts setting, which emphasizes drawing connections between all parts of a student's education.
In such a setting, I would envision building narratives based on real-world applications of CS in many different fields---encompassing not only physical and social sciences, but also arts and humanities (through subfields like Digital Humanities and Library Science).
Doing so will help my CS teaching feel more directly relevant and approachable to students of varied interests, CS majors and non-majors alike.%
}

\newcommand{\gennarrativeend}{%
In addition to helping make the core concepts of CS feel more grounded and approachable, I also believe that teaching with narratives has the additional benefit of introducing students to the many real-world applications of Computer Science, which will benefit both CS majors and non-majors alike.
}

\newcommand{\lainteractionend}{I would also encourage students to help each other by working together in live coding sessions---an approach particularly well-suited to a small classroom setting like that found at \schoolname.}

\newcommand{\geninteractionend}{I would further complement this by encouraging students to work together in small groups for live coding sessions, which can help this technique scale to very large classes.}

\newcommand{\lateachingend}{%
Finally, I also look forward to the opportunity to expand \schoolname's course offerings; in particular, I would be interested in developing a new course on Computational Social Science, a topic that fits well with the interdisciplinary and well-rounded spirit of a Liberal Arts education.
}

\newcommand{\genteachingend}{%
Finally, I also look forward to the opportunity to expand \schoolname's course offerings; in particular, I would be interested in developing a new course on Computational Social Science, an increasingly important field of study that is relevant to work in social media and human-centered computing.
}

\newcommand{\laresearchintro}{%
This research agenda is particularly well suited to a liberal arts institution like \schoolname, giving undergraduate students an opportunity to work with cutting-edge technology and simultaneously study the impact of this technology on society.
}

\newcommand{\genresearchintro}{%
With its focus on real-world applications, this research agenda is particularly well suited for undergraduate involvement, offering undergraduates an opportunity to apply the programming skills they have learned in class to a continuously evolving research setting.
}

\newcommand{\laresearchclosing}{%
Overall, I believe that my research agenda's unique mix of research and engineering components makes it a perfect fit for a primarily-undergraduate institution like \schoolname, and I am excited for the opportunity to continue working with talented undergraduates to pursue this groundbreaking and socially impactful research.
}

\newcommand{\genresearchclosing}{%
Overall, I believe that my research agenda's unique mix of research and engineering components makes it ideally suited as a learning experience for undergraduate students regardless of their post-graduation goals, and I am excited for the opportunity to continue working with talented undergraduates to pursue this groundbreaking and socially impactful research.
}

\newcommand{\deioutreach}{%
For instance, inspired by my experience as a Peer Academic Liaison, I would be interested in pioneering a similar program at \schoolname, either at a whole-college level or as a department level program that could offer more targeted support and advice.%
}

% import the school-specific header to make the magic happen!
% Not an actual school; meant to provide placeholder values
% for school-specific variables for the purposes of compiling
% generic versions of the statements / testing the build chain
\renewcommand{\schoolname}{\textcolor{red}{COLLEGE\_NAME}\xspace} 
\renewcommand{\schoolnamelong}{\textcolor{red}{COLLEGE\_NAME\_LONG}\xspace}
\renewcommand{\schooladdress}{\textcolor{red}{47 Generic St., City, ST 99999}\xspace}
\renewcommand{\schoolintrocourses}{\textcolor{red}{[school-specific intro course numbers]}\xspace}
\renewcommand{\schooladvcourses}{\textcolor{red}{[school-specific AI/ML/NLP course numbers]}\xspace}
\renewcommand{\chairname}{\textcolor{red}{Prof. Search Chair}\xspace}
\renewcommand{\chairlastname}{\textcolor{red}{Prof. Chair}\xspace}
\renewcommand{\chairtitle}{\textcolor{red}{Chair, Faculty Search Committee}\xspace}
\renewcommand{\department}{Department of Computer Science\xspace}
\renewcommand{\position}{\textcolor{red}{POSITION\_NAME (POSITION\_ID)}\xspace}
\longdeitrue
\appendixtrue
\liberalartstrue
%% END CUSTOMIZATION LOGIC

\title{Towards Computational Methods for Assisting in the Proactive Moderation of Online Communities}

\begin{document}
\maketitle

\begin{center}
Research Statement --- Jonathan P. Chang
\end{center}

One of the biggest problems facing online communities today is the prevalence of incivility, harassment, hate speech, and other similar behaviors---collectively referred to in social science as ``antisocial behavior''.
While a lot of Computer Science research has responded to this problem by developing algorithms to detect antisocial behavior, I believe that this approach implicitly centers the perspective of platform owners, who would be in a position to employ these algorithms for moderation, and fails to center an equally important perspective: that of the \emph{communities} of ordinary people who interact on these platforms.
Therefore, my research focuses on the following question: how can technology help online communities to maintain norms and negotiate disputes---in other words, to \emph{proactively} prevent antisocial behavior from taking root?
I tackle this question through a combined social and technical research agenda, which involves engaging collaboratively with real online communities to identify their needs, pioneering new algorithmic methods to address those needs, and finally going back to the communities to collaboratively evaluate these new methods and identify future directions for research.

\section{Background: The Landscape of Moderation}

Popular discourse on content moderation tends to focus on the practices of social media companies like Facebook and Twitter, which adopt a centralized model of moderation where company-employed moderators remove content deemed to be in violation of the platform's (often opaque) rules.
Yet a growing body of scholarship within social science and human-computer interaction argues that moderation actually encompasses a much richer body of practices extending all the way down to the community level \cite{brewer_inclusion_2020,lampe_slashdot_2004,seering_reconsidering_2020}.
Community-driven moderation practices usually involve volunteer moderators who, unlike company-employed moderators, are actual members of the communities they serve, and play a role akin to a community leader.
To this end, rather than relying solely on punitive steps like content removal, they also engage in community-building steps like publicly modeling good behavior, educating community members about the rules, and mediating discussions that are getting out of hand \cite{seering_shaping_2017,billings_understanding_2010}.
Such steps can be seen as \emph{proactive} moderation aimed at discouraging antisocial behavior from occurring at all---in contrast to company-employed moderators' \emph{reactive} practices which acts in response to antisocial behavior after it occurs \cite{lo_when_2018}.

There has been a great deal of interest within the field of Computer Science in developing algorithmic methods to assist in content moderation.
However, prior work has largely catered to the needs to platform-driven moderation, resulting in tools that fit into a reactive moderation workflow---for example, algorithms to automatically detect content that may be antisocial and therefore in need of removal.
By contrast, computational methods for assisting community-driven proactive moderation are a relatively understudied area, and my research aims to fill this gap.


%Leveraging these logs, I conducted a data-driven analysis of how moderators interact with the editors whom they temporarily block, presented at the 2019 Web Conference (WWW 2019) \cite{chang_trajectories_2019}.
%A key finding was the importance of two-way trust in this relationship: blocked editors can signal trust in the moderator by acknowledging their wrongdoing (as measured through handwritten rules that detect actions such as apologies) while moderators can signal trust in the editor by granting requests for reduced block duration.
%Both such strategies are correlated with an increased likelihood of future rule-following by the blocked editor, showing how building trust may be a key factor in proactively fostering a prosocial, norm-adhering culture within an online community.


%These results point to an important aspect of community-driven moderation: it is very much a \emph{collaborative} project, one in which regular community members are not merely passive observers but instead play a direct role.
%Regular community members actively choose to accept---or not---the authority of volunteer moderators and the norms they promote, and may directly communicate their (un)acceptance to the moderators themselves.
%Volunteer moderators, for their part, seek to bolster their relationship of mutual trust with the community; in doing so, they may make community members more willing to adhere to rules and norms, thereby proactively reducing the likelihood of antisocial behavior.
%But how is this concretely achieved?
%My preceding data-driven analysis only reveals one strategy, the granting of clemency, which is applicable only in specific circumstances.
%Gaining a more complete picture of proactive moderation work requires taking a more holistic approach to analyzing the work of volunteer moderators as seen directly from their own perspectives.

%These results illuminate the collaborative nature of community-driven moderation: regular community members are a key contributor to the functioning of the system, as its success requires their buy-in and trust, which moderators in turn must foster.
%But besides grating clemency---which is a highly situational, niche strategy---how do volunteer moderators bolster their relationship with their community and proactively encourage prosocial behavior?
%To answer this question, I conducted interviews with Wikipedia moderators, in work published at the 2022 ACM Conference on Computer-Supported Collaborative Work (CSCW 2022) \cite{schluger_proactive_2022}.
%These interviews focused on the proactive side of volunteer moderators' work, asking questions about moderators reason about the likelihood of a stiuation deteriorating into antisocial behavior, and what kinds of proactive steps they take to prevent such outcomes.
%From the responses, a clear picture emerges of how volunteer moderators can tell when conflicts between Wikipedia editors are at risk of veering away from healthy disagreement and towards antisocial behavior, and the strategies they employ to get things back on track.

%To gain a broader view of proactive moderation strategies, I turned to interviews with Wikipedia moderators, which formed the basis of work presented at the 2022 ACM Conference on Computer-Supported Collaborative Work (CSCW 2022) \cite{schluger_proactive_2022}.
%All moderators in the interviews reported that they have some intuition for when a discussion is at risk of derailing into personal attacks or other prohibited antisocial behaviors.
%Furthermore, they described how when they discover such at-risk discussions, they may intervene to gently steer the conversation back on track, through techniques such as offering to mediate the disagreement.

\section{My Research Contributions}

\subsection{Identifying Community Moderators' Needs}
When developing technology meant to help a particular group or population, it is good practice to hear the perspective of people from that group and let them explain their needs in their own words.
Following this practice, I conducted interviews with community moderators---work that was done in collaboration with undergraduate student Charlotte Schluger and presented at the 2022 ACM Conference on Computer-Supported Collaborative Work (CSCW 2022; a top venue for research on online communities) \cite{schluger_proactive_2022}.
These interviews revealed that volunteer moderators rely on their intuition to tell when a discussion is at risk of later derailing into antisocial behavior---that is, to \emph{forecast} derailment---and will intervene in such discussions to mediate the dispute and keep the conversation on track.
However, they feel challenged by the sheer scale of their communities: they often miss out on opportunities to intervene because there are too many discussions to keep track of; consequently, they express a desire for effective computational tools to aid in this process.
Based on this, we argue that if there existed an algorithm that could forecast derailment much like volunteer moderators do, such an algorithm could ease the burden on volunteer moderators by notifying them of at-risk discussions they may have otherwise missed, or even going further and partly automating the intervention process, e.g., by automatically serving reminders of the rules (much like human moderators currently do) when an at-risk discussion is detected.


\subsection{Demonstrating the Feasibility of Algorithmically Forecasting Derailment}
As discussed above, an algorithm capable of forecasting future derailment in online discussions could form the basis of a computational tool to assist volunteer moderators.
To demonstrate that such forecasting is a feasible task for computers, I conducted two experiments in collaboration with industry partners, both of which start from socio-linguistic theories about what factors within discussions might be associated with tension and conflict, then aim to experimentally evaluate whether algorithmic estimates of these factors are correlated with future derailment.
The first of these works, conducted together with researchers from Google and the Wikimedia Foundation, examines the role of impoliteness.
The second, conducted as an internship project at Facebook, examines the role of misperceptions of a speaker's intent.
In both cases we found that when the factor being studied (impoliteness or misperception) was present early in a discussion, as measured by machine learning models, the discussion was at statistically significantly higher risk of future derailment into antisocial behavior.
These results were respectively presented at the 2018 Annual Meeting of the Association for Computational Linguistics (ACL 2018) \cite{zhang_conversations_2018} and the 2020 Web Conference (WWW 2020) \cite{chang_dont_2020}, both of which are top venues for research on natural language processing and computational social science.

\subsection{Practical Forecasting of Conversational Derailment}
Classical machine learning approaches to natural language processing (like those used in my earlier proof-of-concept work) typically operate on a fixed snapshot of a discussion, assuming it will never change.
But such an approach is impractical for forecasting derailment in real moderation settings, because online discussions actually evolve over time, and new comments that are posted might increase or decrease the risk of derailment.
Thus, a practically useful algorithm would need to follow a conversation in real time, and at each new comment update its belief about whether the conversation is at risk of derailment.
In work presented at the ACL's 2019 Conference on Empirical Methods in Natural Language Processing (EMNLP 2019; a venue commonly used to debut novel natural language models) \cite{chang_trouble_2019}, I was able to develop such an algorithm, a neural network known as CRAFT, by adapting techniques from the domain of dialog modeling (aka ``chatbots'') \cite{serban_building_2016}, where similar practical challenges are present.
Experiments on conversations sampled from real online communities show that 77\% of discussions that actually derail are correctly detected by CRAFT (this is a standard machine learning evaluation metric known as recall); conversely, across all conversations where CRAFT predicts derailment, 64\% in fact go on to derail (a standard metric known as precision).

\subsection{Using Forecasting Algorithms to Support Online Communities}
While CRAFT scores highly on standard machine learning metrics, these metrics do not necessarily translate into actual usefulness for humans.
Evaluating whether a forecasting algorithm like CRAFT can actually benefit real online communities required bringing together the social and technical aspects of my work.
On the techical side, I led a team of undergraduate and masters students to develop a prototype browser extension, ConvoWizard, that uses CRAFT to warn users when a discussion thread they are replying to may be at risk of derailment, and also give gives them feedback as they draft their reply, warning them if their reply might make the situation even more tense.
On the social side, I arranged a groundbreaking collaboration with a real online community, in which we worked together with the moderators to conduct a real-world test run of ConvoWizard in a way that was respectful of the community's norms.

This test run produced a number of promising findings that suggest ConvoWizard is useful to humans and can improve the quality of online discussions.
Upon seeing warnings from ConvoWizard, users tended to make edits that decrease the risk of derailment (as estimated by CRAFT), unlike in the control condition where edits tended to increase the risk.
A linguistic analysis of the resulting comments showed that they tend to exhibit increased use of known conflict avoidance strategies such as asking questions and adopting more formal tone.
Finally, in an exit survey conducted after the test run, ConvoWizard users left largely positive evaluations of the tool: most notably, 54\% reporting that ConvoWizard made them rethink posting a comment they might have later regretted, and 63.8\% felt that if ConvoWizard were deployed at scale, the net effect would be an improvement in discussion quality.
These and other findings from the ConvoWizard test run were presented at CSCW 2022 \cite{chang_thread_2022}.

\section{Next Steps}
I am a firm believer in the principle that before new technologies can be broadly adopted, we must first try to understand as fully as possible its ramifications for society and its potential risks.
This is especially vital for work relating to content moderation, which is quickly becoming a major societal issue with implications for online safety, free speech, and social justice.
The social aspect of my work has been tailored towards understanding the impact of tools based on forecasting algorithms, and while I regard the findings from my interviews and the ConvoWizard user study as encouraging positive signs, we are still far from fully understanding the effects of these tools, and thus there is much work still to do.

One area that I particularly want to focus on is improving the transparency of forecasting algorithms.
My current approach, CRAFT, uses a neural network, a methodology whose decision processes are notoriously hard to explain.
Indeed, lack of transparency was the biggest complaint we received in the ConvoWizard exit survey, with users reporting that a warning about risk of derailment, with no explanation, is difficult to act upon even if they agree with it.
I regard this as the current biggest \emph{technical} hurdle to the usability of forecasting algorithms, and have already begun work on incorporating the latest research on explainable neural models to build a more transparent successor to CRAFT.
But this is not just short-term work: as the field of ``explainable AI'' grows, it will undoubtedly produce more breakthroughs that we can draw from, and I remain committed to pursuing this goal because I believe that if forecasting-based tools are ever broadly adopted, they must be based on explainable algorithms.

But the transparency issue is not just a technical problem: it also ties in to the larger \emph{social} problem of algorithmic bias.
By now, it has become well known that data-driven approaches are vulnerable to capturing biases embedded within the data.
The question, then, is not whether forecasting algorithms like CRAFT contain biases---the answer is surely yes---but rather what kinds of biases exist and what their impact would be in a broadly deployed forecasting-based tool.
This is uncharted territory, since most prior work on computational tools for moderation---and by extension, the analyses of their biases---focused on the reactive paradigm, and it is not clear how those findings might apply to proactive moderation.
Like the rest of my research agenda, my plan for addressing this question is to take a combined social and technical approach.
Once an explainable successor to CRAFT has been developed, my plan is to analyze the model's decisions through the lens of recent work from computational social science on codifying the social and power implications found in natural language \cite{sap_social_2020}.
Cross-referencing this work with the model's decision-making processes could reveal, for instance, whether the algorithm is less sensitive to attacks towards specific underrepresented groups.
And once such biases are identified, the same insights from social science could be applied to design ways to counter them.

It is also important once again to understand the impact of tools based on forecasting algorithms not just in the lab, but in the context of real communities.
All the future plans I just described should therefore be accompanied by further collaborative studies with real communities, including much larger-scale studies aimed at understanding long-term impacts, not just immediate implications on individual discussion threads.
As described earlier, these studies will also look into how forecasting algorithms might be used to directly support moderators, in additional to ConvoWizard-style tools aimed at regular community members.

\section{Involving Undergraduates in my Research Agenda}
Throughout my research, I have had the pleasure and privilege of collaborating with several undergraduate students, who have contributed to my research agenda in a number of ways.
Most notably, two of my papers---the two CSCW papers---were co-authored with undergraduate student Charlotte Schluger, who is in fact first author on one of them.
She had the distinction of contributing through the entire research pipeline: she was part of the larger student team that implemented the ConvoWizard research tool, was responsible for both designing and conducting the moderator interviews, and finally contributed large portions of the paper writing.
To me, Charlotte's experience illustrates the broad potential of my research agenda for undergraduates: it presents opportunities for both scientific contributions and software engineering contributions, thus catering to undergraduate students on any career path; indeed, other members of the ConvoWizard development team, knowing they wanted an industry job, chose to focus solely on the software engineering aspects.

My research agenda also opens up other paths for undergraduates to contribute beyond just work directly aimed at publications.
I have a strong commitment to open source and open science, and in this spirit I have been a core contributor to ConvoKit (\url{https://convokit.cornell.edu}), an open source Python package designed to empower research on conversational data.
As part of my work on ConvoKit I have mentored a number of undergraduate contributors, who again have gone onto a variety of career paths.
In fact, I met my first undergraduate mentee, Andrew Wang, through his work on ConvoKit, which he used as a launching point to get into research; he would later go on to receive a MS degree at Stanford followed by returning to Cornell to pursue a PhD.
On a different type of career path is Caleb Chiam, who over his 3 years working with us as an undergraduate became one of the biggest contributors to ConvoKit---experience which now serves him well at his current software engineering career.

Overall, I believe that my research agenda is a perfect fit for an undergraduate teaching institution.
Because of my focus on practical use cases, my research involves a heavy software engineering focus alongside the typical scientific aspects.
As my prior experiences with undergraduate students have shown, this means that my research has something to offer for everyone---a particularly important feature at an undergraduate institution, where students may be pursuing a variety of different post-graduation trajectories, or in fact trying to decide on one.
My continued commitment to open source and the ConvoKit project also offers a unique opportunity for students to have an impact on a real tool that is used by researchers and software engineers while simultaneously getting exposure to research.
In light of all of this, I feel that I can strongly contribute to the education of students at UNIVERSITY\_NAME.

\vspace{\baselineskip}
\bibliographystyle{plain}
\bibliography{refs}

\end{document}
