\documentclass[12pt,letterpaper]{article}
\usepackage[margin=0.75in]{geometry}
\usepackage{graphicx}
%\usepackage[bf,tiny,compact]{titlesec}
\usepackage{times}
\usepackage{xcolor}
\usepackage{xspace}
\usepackage{enumerate}
\usepackage{enumitem}
\usepackage{titlesec}
\usepackage{etoolbox}
\usepackage{hyperref}


\setlist{nosep}

\makeatletter
\def\@maketitle{%
    {\centering\fontsize{12pt}{14pt}\selectfont\textbf{\@title}\par}
}
\patchcmd{\@footnotetext}{\footnotesize}{\fontsize{10pt}{12pt}\selectfont}{}{}
\makeatother

\titleformat{\section}[runin]{\normalfont\normalsize\bfseries}{}{0em}{}[.]
\titlespacing*{\section}{0em}{0.25\baselineskip}{0.5em}

%% LOGIC FOR CUSTOMIZING THE STATEMENT FOR INDIVIDUAL SCHOOLS
% set up common defines (commands and boolean flags)
% Sets up custom commands and flags for school-specific behaviors
% Also assigns default values, but these can be overwritten with school-specific
% values in the corresponding header file.
% Some default values are placeholders (marked in red) and these MUST be 
% overwritten in the school-specific headers!
\newcommand{\schoolname}{\textcolor{red}{COLLEGE\_NAME}\xspace} 
\newcommand{\schoolnamelong}{\textcolor{red}{COLLEGE\_NAME\_LONG}\xspace}
\newcommand{\schooladdress}{\textcolor{red}{47 Generic St., City, ST 99999}\xspace}
\newcommand{\schoolintrocourses}{\textcolor{red}{[school-specific intro course numbers]}\xspace}
\newcommand{\schooladvcourses}{\textcolor{red}{[school-specific AI/ML/NLP course numbers]}\xspace}
\newcommand{\chairname}{\textcolor{red}{Prof. Search Chair}\xspace}
\newcommand{\chairlastname}{\textcolor{red}{Prof. Chair}\xspace}
\newcommand{\chairtitle}{\textcolor{red}{Chair, Faculty Search Committee}\xspace}
\newcommand{\department}{Department of Computer Science\xspace}
\newcommand{\position}{\textcolor{red}{POSITION\_NAME (POSITION\_ID)}\xspace}

\newif\iflongdei
\longdeitrue 
\newif\ifappendix
\appendixtrue 
\newif\ifliberalarts
\liberalartstrue
\newif\ifsupplementals
\supplementalsfalse

\newcommand{\materials}{my latest CV, a graduate transcript, and statements on teaching, research, and commitment to diversity and inclusion.\xspace}

\newcommand{\coverteachingpara}{%
\textcolor{red}{[school-specific segment on teaching philosophy]}
}

\newcommand{\coverresearchpara}{%
\textcolor{red}{[school-specific segment on research]}
}

\newcommand{\lateachingintro}{%
Liberal Arts education played a formative role in my approach to teaching: my love of teaching was first sparked during my undergraduate studies at Harvey Mudd College.
That was where I got my first (small) taste of what it is like to be an instructor, through my participation in the CS department's ``grutoring'' program---similar to TA positions found at larger universities, but with more emphasis on direct interaction with students.
This early experience of engaging directly with individual students, and seeing the specific things they struggled with, strongly influenced how I think about teaching.
To this day, I approach teaching by starting from a student's perspective: for any given course, what is a student most likely to struggle with, and what's the most effective way to help them understand it?
While the specific answers to these questions will vary with each course and each student cohort, there are nevertheless a few concrete strategies which I have found to be good \emph{starting points} for any course: \textbf{teaching with narratives}, \textbf{leveraging technology for interactivity}, and \textbf{building inclusive learning environments}.%
}

\newcommand{\genteachingintro}{%
My love of teaching was first sparked during my undergraduate studies at Harvey Mudd College.
That was where I got my first (small) taste of what it is like to be an instructor, through my participation in the CS department's ``grutoring'' program---similar to TA positions found at larger universities, but with more emphasis on direct interaction with students.
This early experience of engaging directly with individual students, and seeing the specific things they struggled with, strongly influenced how I think about teaching.
Even as I have moved on to teaching in larger classroom settings during my graduate career, I continue to approach teaching by starting from a student's perspective: for any given course, what is a student most likely to struggle with, and what's the most effective way to help them understand it?
While the specific answers to these questions will vary with each course and each student cohort, there are nevertheless a few concrete strategies which I have found to be good \emph{starting points} for any course: \textbf{teaching with narratives}, \textbf{leveraging technology for interactivity}, and \textbf{building inclusive learning environments}.%
}

\newcommand{\lanarrativeend}{%
This makes it particularly valuable for a Liberal Arts setting, which emphasizes drawing connections between all parts of a student's education.
In such a setting, I would envision building narratives based on real-world applications of CS in many different fields---encompassing not only physical and social sciences, but also arts and humanities (through subfields like Digital Humanities and Library Science).
Doing so will help my CS teaching feel more directly relevant and approachable to students of varied interests, CS majors and non-majors alike.%
}

\newcommand{\gennarrativeend}{%
In addition to helping make the core concepts of CS feel more grounded and approachable, I also believe that teaching with narratives has the additional benefit of introducing students to the many real-world applications of Computer Science, which will benefit both CS majors and non-majors alike.
}

\newcommand{\lainteractionend}{I would also encourage students to help each other by working together in live coding sessions---an approach particularly well-suited to a small classroom setting like that found at \schoolname.}

\newcommand{\geninteractionend}{I would further complement this by encouraging students to work together in small groups for live coding sessions, which can help this technique scale to very large classes.}

\newcommand{\lateachingend}{%
Finally, I also look forward to the opportunity to expand \schoolname's course offerings; in particular, I would be interested in developing a new course on Computational Social Science, a topic that fits well with the interdisciplinary and well-rounded spirit of a Liberal Arts education.
}

\newcommand{\genteachingend}{%
Finally, I also look forward to the opportunity to expand \schoolname's course offerings; in particular, I would be interested in developing a new course on Computational Social Science, an increasingly important field of study that is relevant to work in social media and human-centered computing.
}

\newcommand{\laresearchintro}{%
This research agenda is particularly well suited to a liberal arts institution like \schoolname, giving undergraduate students an opportunity to work with cutting-edge technology and simultaneously study the impact of this technology on society.
}

\newcommand{\genresearchintro}{%
With its focus on real-world applications, this research agenda is particularly well suited for undergraduate involvement, offering undergraduates an opportunity to apply the programming skills they have learned in class to a continuously evolving research setting.
}

\newcommand{\laresearchclosing}{%
Overall, I believe that my research agenda's unique mix of research and engineering components makes it a perfect fit for a primarily-undergraduate institution like \schoolname, and I am excited for the opportunity to continue working with talented undergraduates to pursue this groundbreaking and socially impactful research.
}

\newcommand{\genresearchclosing}{%
Overall, I believe that my research agenda's unique mix of research and engineering components makes it ideally suited as a learning experience for undergraduate students regardless of their post-graduation goals, and I am excited for the opportunity to continue working with talented undergraduates to pursue this groundbreaking and socially impactful research.
}

\newcommand{\deioutreach}{%
For instance, inspired by my experience as a Peer Academic Liaison, I would be interested in pioneering a similar program at \schoolname, either at a whole-college level or as a department level program that could offer more targeted support and advice.%
}

% import the school-specific header to make the magic happen!
% Not an actual school; meant to provide placeholder values
% for school-specific variables for the purposes of compiling
% generic versions of the statements / testing the build chain
\renewcommand{\schoolname}{\textcolor{red}{COLLEGE\_NAME}\xspace} 
\renewcommand{\schoolnamelong}{\textcolor{red}{COLLEGE\_NAME\_LONG}\xspace}
\renewcommand{\schooladdress}{\textcolor{red}{47 Generic St., City, ST 99999}\xspace}
\renewcommand{\schoolintrocourses}{\textcolor{red}{[school-specific intro course numbers]}\xspace}
\renewcommand{\schooladvcourses}{\textcolor{red}{[school-specific AI/ML/NLP course numbers]}\xspace}
\renewcommand{\chairname}{\textcolor{red}{Prof. Search Chair}\xspace}
\renewcommand{\chairlastname}{\textcolor{red}{Prof. Chair}\xspace}
\renewcommand{\chairtitle}{\textcolor{red}{Chair, Faculty Search Committee}\xspace}
\renewcommand{\department}{Department of Computer Science\xspace}
\renewcommand{\position}{\textcolor{red}{POSITION\_NAME (POSITION\_ID)}\xspace}
\longdeitrue
\appendixtrue
\liberalartstrue
%% END CUSTOMIZATION LOGIC

\title{SOME FANCY TITLE HERE}

\begin{document}
\maketitle

{\centering Teaching Statement --- Jonathan P. Chang \par}

\vspace{0.5\baselineskip}
% TODO different opening paragraph for non liberal arts colleges
Liberal arts education played a formative role in my approach to teaching---my love of teaching was first sparked during my undergraduate studies at Harvey Mudd College.
That was where I got my first (small) taste of what it is like to be an instructor, through my participation in the CS department's ``grutoring'' program---similar to TA positions found at larger universities, but with more emphasis on direct interaction with students.
I quickly fell in love with the rewarding experience of working with students to help them understand challenging concepts, which eventually led my interest in a teaching career.

Through my subsequent teaching experiences, I have developed a guiding philosophy for my teaching: I aim for learning environments where students can feel a sense of \emph{ownership} and \emph{accomplishment} over their learning.
Students in my classroom should not feel like mere receivers of knowledge, but instead should feel like active participants working together with me and each other to arrive at a shared understanding of the material.
During my PhD, I have had the chance to implement this philosophy in two disparate settings: a more traditional classroom setting as co-instructor of record for CS4300 ``Information and Language'',\footnote{\url{https://classes.cornell.edu/browse/roster/SP23/class/CS/4300}} and a more ad-hoc setting as a designer and instructor of data science workshops for the Cornell Center for Social Sciences (CCSS).\footnote{\url{https://socialsciences.cornell.edu/computing-and-data/workshops-and-training}}
These experiences serve to illustrate three concrete strategies that emerge from my teaching philosophy: \textbf{teaching with narratives}, \textbf{leveraging technology for interactivity}, and \textbf{building inclusive learning environments}.

\section{Teaching With Narratives}
One way I help students feel like a course is made \emph{for them} is to ground course content in \emph{narratives} that are relevant to their interests.
This is particularly helpful for reaching students whose backgrounds and interests are outside of Computer Science---for example, when I was developing an Introduction to Python workshop for CCSS, which was meant for a target audience of social science students who may have never programmed before, my approach was to construct a narrative around analyzing a real social science dataset from CCSS, grounding the core concepts of programming in steps a researcher would need to take in preparing and analyzing the data for social science research.
For example, the need to store and represent datapoints motivates the concept of variables, and the need to repeatedly perform common operations across all the data motivates the concepts of loops and functions.
Student feedback indicates that this approach was an effective and approachable way to communicate key concepts in programming:
\begin{quote}
    This is one of the best tutorials I've been to. The instructors were knowledgeable, kind, and better at communicating/explaining code then most I've seen.
\end{quote}

I also leveraged this narrative-based approach in my CS4300 teaching, for instance by drawing connections between lecture topics and recent advances in language technology that students may have heard about in the news.
Students seemed to recognize my efforts: on the course evaluation question ``Did the lecturer motivate the course content
and place it in the context of your major or your overall engineering
education?'' students rated me an average 4.12 out of 5.
Looking ahead, I believe this strategy will be also very applicable for teaching undergraduate intro-level CS courses.
\narrativeendsent

\section{Leveraging Technology for Interactivity}
While collaboration technology is often associated with remote learning and flipped classrooms, I believe it also has great potential to enrich traditional classroom environments, making lectures more interactive so students can feel actively engaged in their learning.
In collaboration with Cornell's Active Learning Institute (ALI), my teaching in CS4300 made heavy use of Google Colaboratory as a shared interactive coding environment.
Using this tool, I integrated live coding into my lectures under a ``copiloting'' paradigm in which I would write parts of the code and ask for student contributions on other parts, resulting in a jointly authored final result that students could feel they concretely contributed to.
I also leveraged online forms and polling alongside old-fashioned pen-and-paper handouts to implement similar collaborative activites in non-coding parts of the lectures as well.
Comments from an ALI-administered survey show how students perceived the benefits of these activities:

\begin{quote}
    [Live coding] helped me learn and be more engaged because I was actively seeing concepts play out in real time.
\end{quote}
\begin{quote}
    I thought they [interactive activities] were great to discuss my thoughts and tinker with our learning with classmates.
\end{quote}

At the same time, other comments point to shortcomings in this approach that I have concrete plans to address in my future teaching.
In particular, some students remarked that live coding could get hard to follow, since falling behind early on would leave them out-of-sync throughout the rest of the live coding session.
To address this, I plan to adapt techniques from my CCSS workshop teaching, such as having designated ``checkpoints'' throughout live coding sessions to give students a chance to catch up, during which an instructor
\ifliberalarts
%
\else
or TA
\fi
could reach out to individual students who may be particularly stuck on some part.
\interactionimprove

\section{Developing Course Content With Diverse Student Backgrounds in Mind}
As I've discussed, my core philosophy holds that students should feel like they are actively working together in their learning.
For this to work, every student must feel like they have a place and are able to contribute, regardless of their background.
To this end, I seek to develop my course content in ways that account for the fact that not every student has had the same opportunities in the past, with support systems to ensure that everyone can compete on the same level.

One major contribution I made in pursuit of this goal as co-instructor of CS4300 was my role in redeveloping the course project requirement.
Historically, the final project has been meant as an opportunity for students to showcase what they have learned by developing a novel information retrieval app.
% TODO check accuracy of this sentence with Cristian
While students have consistently found this to be one of the most exciting aspects of the class, one downside is that students who have had the privilege of prior app development experience---e.g., through internships or coding bootcamps---may find themselves with a head start compared to those who have not, as the latter group would need to first acquaint themselves with the basics of building an app before they can even get to thinking about the information retrieval aspect, which is where the actual intended educational value lies.
To address this, I oversaw the design and deployment of a new project server interface, which is designed to automatically handle the boilerplate aspects of app development/deployment (e.g., setting up a database connection) as well as a common starter template for all students to use, which automatically handles the work of laying out a user interface.

While the deployment of this new platform did not go perfectly smoothly---as with any new software, there ended up being a number of bugs that sometimes caused inconveniences and outages---student feedback neverthless suggests that it did help achieve my goal of empowering students to succeed and thrive on the project regardless of background experience.
One comment we received specifically called this out: ``I think this [project server] was really nice as it really leveled the playing field with regards to prior experience.''
Other comments even suggested that with this system in place, the final project was even able to serve as an accessible introduction to app development; for instance, `` I feel that it added to my knowledge about front-end and back-end programming.''

The project server goes hand in hand with other, less ambitious but still important, course design choices aimed at accounting for diverse student backgrounds and needs.
\iflongdei
I discuss these in more detail in my statement on diversity, equity, and inclusion.
\else
For instance, in recognition of the fact that not all students may be comfortable vocally contributing in class, I sought to develop alternative avenues of participation---such as the poll-based activities described previously.
\fi
Of course, I recognize that these constitute only small steps towards achieving a more equitable and inclusive classroom.
I look forward to continuing to search for ways to design a learning environment where anyone can thrive, and I am excited for the chance to learn from other faculty and students at \schoolname so I can incorporate and build upon the work they have done in this direction.

\section{Future Teaching Plans}
I have only just gotten started on my teaching journey.
As I look ahead to hopefully a new chapter at \schoolname, I am especially excited for the opportunity to teach some of the foundational courses, most notably \textcolor{red}{TODO school-specific course numbers will go here}.
While I have taught at a variety of levels throughout my prior career, foundational CS courses are what excite me the most---it is deeply satisfying for me to be able to successfully convey the core concepts of computer science, and the fact that these concepts are still novel to the students always makes for interesting conversations that may even lead me to see old ideas in a new light.
That said, I am still also passionate about teaching higher-level courses related to the research agenda I have laid out in my research statement, such as \textcolor{red}{TODO school-specific AI/ML/NLP course numbers}.
Finally, I also look forward to the opportunity to expand \schoolname's course offerings; in particular, I would be interested in developing a new course on computational social science---lying at the intersection of STEM and the humanities, I believe such a course would be a perfect fit for a computer science department at a liberal arts institution.

\ifappendix
\vspace{\baselineskip}
\section{Appendix: Selected Comments From Student Evaluations}
Below are some selected comments from student evaluations and surveys which did not fit in any particular section above, but which I felt are otherwise indicative of my teaching philosophy and its outcomes.
Comments are grouped by source.
The source data---that is, the evaluations themselves in the raw format they were provided to me---is available in the supplementary materials attached to my submission.

%\noindent\underline{CS4300 ALI survey}

\noindent\underline{CS4300 midterm evaluations}
\begin{itemize}
    \item I really enjoy Jonathan's teaching, he's very passionate when he speaks which keeps me engaged.
    \item Great at engaging with the audience and working to keep attention
\end{itemize}

\noindent\underline{CS4300 end-of-semester evaluations}
\begin{itemize}
    \item Made concepts much easier to understand and was engaging during the lectures, and was super helpful during his office hours
    \item Very engaging and well-spoken. Learned the most from Professor Chang's lectures. I thought the use of the active note taking assisted in keeping me engaged and taking useful notes.
    \item Genuinely. Prof. Chang was great at teaching, so his lectures were pretty helpful
\end{itemize}

\noindent\underline{CCSS workshops}\\
\emph{[Note: comments are taken from a survey question about my Introduction to Python workshop; references to ``this'' and ``it'' are referring to that workshop.]}
\begin{itemize}
    \item Thank you...I finally understood how to use Python. I have taken other classes at Cornell before but none were so comprehensive as yours.
    \item It was a great introduction, friendly, and overall positive environment. It also was taught extremely well, at the perfect pace and everything was explained very clearly. It also gave students the opportunity to get to know instructors and each other. The basics were taught very well.
\end{itemize}

\else
%
\fi

\end{document}
