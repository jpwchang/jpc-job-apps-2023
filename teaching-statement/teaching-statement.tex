\documentclass[11pt,letterpaper]{article}
\usepackage[margin=0.75in]{geometry}
\usepackage{graphicx}
%\usepackage[bf,tiny,compact]{titlesec}
\usepackage{times}
\usepackage{cite}
\usepackage{natbib}
\usepackage{titlesec}
\usepackage{etoolbox}
\usepackage{hyperref}


\makeatletter
\def\@maketitle{%
    {\centering\fontsize{12pt}{14pt}\selectfont\textbf{\@title}\par}
}
\patchcmd{\@footnotetext}{\footnotesize}{\fontsize{10pt}{12pt}\selectfont}{}{}
\makeatother


\renewcommand{\bibsection}{\noindent\textbf{References (works to which I contrubted have my name \underline{underlined})}\vspace{-6pt}}
\setlength{\bibsep}{0pt} % or use whatever dimension you want
\renewcommand{\bibfont}{\fontsize{10pt}{12pt}\selectfont} % or any other  appropriate font command


\renewcommand{\section}[1]{\vspace{0.25\baselineskip}\noindent\textbf{#1.}}
\renewcommand{\subsection}[1]{\vspace{0.25\baselineskip}\noindent\textit{#1.}}

\title{SOME FANCY TITLE HERE}

\begin{document}
\maketitle

\begin{center}
Teaching Statement --- Jonathan P. Chang
\end{center}

% TODO different opening paragraph for non liberal arts colleges
Liberal arts education played a formative role in my approach to teaching---my love of teaching was first sparked at Harvey Mudd College, where I did my undergraduate degree.
Specifically, Harvey Mudd was where I got my first small taste of what it is like to be an instructor, through my participation in the CS department's ``grutoring'' program---similar to the TA positions found at larger universities, but with a greater emphasis on direct interaction with students.
Though I initially joined the program just out of curiosity (and admittedly also for the chance to get paid!), I quickly fell in love with both the challenge of finding the best way to explain a difficult concept to a student, and the satisfaction of seeing the ``aha'' moment where they finally get it.
Since then, my later experiences as an actual instructor have been spent chasing those ``aha'' moments---and more concretely, developing and implementing a strategy to make those moments happen more often and reliably.

Subsequently, I have developed a guiding philosophy for my teaching: I aim for learning environments where students can feel a sense of \emph{ownership} and \emph{accomplishment} over their learning.
Students in my classroom should not feel like mere receivers of knowledge, but instead should feel like active participants working together with me and with each other to arrive at a shared understanding of the material.
During my PhD, I have had the chance to implement this philosophy in two disparate settings: a more traditional classroom setting as co-instructor of CS4300 ``Information and Language'', and a more personal and ad-hoc setting as a designer and instructor of data science workshops for the Cornell Center for Social Sciences (CCSS).
These experiences serve to illustrate three concrete strategies that emerge from my teaching philosophy: \textbf{narrative framings for lesson plans}, \textbf{technological innovations for interactivity}, and \textbf{developing curricula with diverse student backgrounds in mind}.

\section{Narrative Framings for Lesson Plans}
Reflecting on my own undergraduate student experience, I have found that the classes I enjoyed and learned from the most were ones where the instructor was able to illustrate a clear narrative underlying the course material.
In other words, rather than simply focusing on individual concepts or proofs, these classes started with a clear story or unifying theme, with each new concept introduced in the class forming a logical progression within that story or theme.

As a concrete example, consider a introductory data structures class.
A narrative framing for such a class could start by giving a big picture overview of the different data organization needs faced by real-world applications, perhaps pointing to existing well-known software for specific examples.
Then, the instructor could call back to this unifying theme when introducing each new data structure, asking students to imagine what applications this data structure might be well-suited for and poorly-suited for, as well as how this data structure might address certain limitations of other data structures previously introduced in the class.
I believe this approach encourages deeper engagement with the course material: by observing ways in which each lesson makes progress towards the goal and ways it still falls short, students can start to imagine how they might approach extending the class concepts to close the remaining gap (tying in to my underlying philosophy of feeling ownership over their learning), and compare/contrast this with the solutions that end up being presented later in the class.

The clearest illustration of how I have implemented this strategy in practice is the Intro to Python workshop that I developed for CCSS.
Like most CCSS workshops, this 4-part bootcamp was designed for a target audience of social science researchers with little to no technical background, who may have never done any programming before.
Explaining core programming concepts---like variables, functions, and logic---to first-time programmers is known to be a challenging problem, and in this case I addressed this challenge by grounding those abstract concepts in a social science narrative: using a real dataset sourced from my CCSS co-instructor, we asked students to imagine that they were using this data for their own research, then framed the lessons in terms of steps a researcher would need to take in preparing and analyzing the data.
For example, 

\end{document}
