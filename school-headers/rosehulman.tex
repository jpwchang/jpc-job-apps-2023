\renewcommand{\schoolname}{Rose-Hulman\xspace}
\renewcommand{\schoolnamelong}{Rose-Hulman Institute of Technology\xspace}
\renewcommand{\schooladdress}{5500 Wabash Avenue, Terre Haute, IN 47803\xspace}
\renewcommand{\schoolintrocourses}{CSSE 120, 220, and 230\xspace}
\renewcommand{\schooladvcourses}{CSSE 286: Introduction to Machine Learning and CSSE 413: Artificial Intelligence}
\renewcommand{\chairname}{Amanda Stouder\xspace}
\renewcommand{\chairlastname}{Prof. Stouder\xspace}
\renewcommand{\chairtitle}{Chair, Faculty Search Committee\xspace}
\renewcommand{\department}{Department of Computer Science \& Software Engineering\xspace}
\renewcommand{\position}{Tenure-Track Faculty in Computer Science and Software Engineering (Requisition 497)\xspace}
\longdeitrue
\appendixtrue
\liberalartsfalse

\renewcommand{\materials}{my latest CV and statements on teaching, research, and commitment to diversity, equity, and inclusion.\xspace}

\renewcommand{\coverteachingpara}{%
I am drawn to Rose-Hulman because I know that, as another small undergraduate-focused institution, it shares this core principle.
When it comes to teaching, I am most passionate about foundational CS courses (like \schoolintrocourses).
Of course, I would also be excited to teach more advanced courses connected to my research interests, such as \schooladvcourses.
Finally, I would love the chance to expand Rose-Hulman's course offerings through new courses in my fields of research: Natural Language Processing and Computational Social Science.%
}

\renewcommand{\coverresearchpara}{%
I believe this research agenda fits well with Rose-Hulman's multidisciplinary approach to Computer Science education.%
}

\renewcommand{\genteachingintro}{%
My education at a primarily-undergraduate institution played a formative role in my approach to teaching---my love of teaching was first sparked at Harvey Mudd College.
That was where I got my first (small) taste of what it is like to be an instructor, through my participation in the CS department's ``grutoring'' program---similar to TA positions found at larger universities, but with more emphasis on direct interaction with students.
This early experience of engaging directly with individual students, and seeing the specific things they struggled with, strongly influenced how I think about teaching.
To this day, I approach teaching by starting from a student's perspective: for any given course, what is a student most likely to struggle with, and what's the most effective way to help them understand it?
While the specific answers to these questions will vary with each course and each student cohort, there are nevertheless a few concrete strategies which I have found to be good \emph{starting points} for any course: \textbf{teaching with narratives}, \textbf{leveraging technology for interactivity}, and \textbf{building inclusive learning environments}.%
}

\renewcommand{\geninteractionend}{This solution might work particularly well in a small classroom setting like that found at \schoolname.}

\renewcommand{\genresearchintro}{%
This research agenda is well suited to an undergraduate-focused institution like \schoolname, giving undergraduate students an opportunity to work with cutting-edge technology and simultaneously study the impact of this technology on society.
}

\renewcommand{\genresearchclosing}{%
Overall, I believe that my research agenda's unique mix of research and engineering components makes it a perfect fit for a primarily-undergraduate institution like \schoolname, and I am excited for the opportunity to continue working with talented undergraduates to pursue this groundbreaking and socially impactful research.
}